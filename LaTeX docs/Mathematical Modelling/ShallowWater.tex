\documentclass{article}
\usepackage{graphicx}
\usepackage{amsmath}
\begin{document}
Tameez Latib

Problem 14

I, Tameez Latib, declare that this work is my own. I did this work honestly and can fully stand behind everything that I have written.

----

Consider an incompressible, inviscid, irrotational flow of a fluid. Suppose that the fluid flows over a flat surface z = 0 where its height is given by

 h(x, t). The length scales are L (horizontally) and A (vertically), and if

$\vec{u}(x, z, t) = (u_1, u_2)$ is the velocity, ($u_1$ is the x velocity, $u_2$ is the z velocity) then $u_1 \approx \sqrt{gA}$  

By irrotational and the Euler equations, 

$$\vec{\bigtriangledown} \times \vec{u} = \vec{0} $$

$$\vec{\bigtriangledown} \cdot \vec{u} = 0 $$

$$\vec{u}_t + \vec{\bigtriangledown} P/{\rho}  = \vec{g} $$

From the second equation, 

$u_{1x} = 0$ (x velocity has no x dependence)

$u_{2z} = 0$ (z velocity has no z dependence) 

Now going to the first equation, 

$u_{2x} = u_{1z}$

Writing out the third equation in component form, 

$${u}_{1t} +  P_x/{\rho}  = 0   $$

$${u}_{2t} +  P_z/{\rho}  = -g$$

To non-dimensionalise this model, 

$x = Lx^*$ (The overall horizontal scale)

$z = Az^*$ (The overall height scale)

$u_1 = \sqrt{gA} u_1^*$  (Since that is what it is comparable to)

$u_2 = \sqrt{gA} u_2^*$  

$P = \bar{P} P^*$  

$t = \bar{t} t^*$  

$h =  \bar{h}h^*$  

Then: 

$$ \frac{\sqrt{gA}}{\bar{t}} {u^*}_{1t^*} +  \frac{\bar{P}}{L \rho}P^*_{x^*}  = 0   $$

$$ \frac{\sqrt{gA}} {\bar{t}} {u^*}_{2t^*} +  \frac{\bar{P}}{A \rho}P^*_{z^*}  = -g   $$

From the first of these two: 

$$ {u^*}_{1t^*} +  \frac{\bar{P}\bar{t}}{L \rho \sqrt{gA}}P^*_{x^*}  = 0   $$

From the second: 

$$ {u^*}_{2t^*} +  \frac{\bar{P}\bar{t}}{A \rho \sqrt{gA}}P^*_{z^*}  = - \frac{g \bar{t}}{ \sqrt{gA}}   $$

So letting

$$\frac{\bar{P}\bar{t}}{L \rho \sqrt{gA}} = 1$$

From the right hand side of the second equation, it makes sense to let  

$$\bar{t} = \frac{L}{\sqrt{gA}}$$

Which gets rid of g, and so this means 

$$\bar{P} = {\rho g A}$$

By this, 

$$ {u^*}_{1t^*} +  P^*_{x^*}  = 0   $$

$$ {u^*}_{2t^*} +  \frac{L}{A}P^*_{z^*}  = - \frac{L}{A}   $$

Now if we let $\epsilon = A/L$ as $A << L$, then We can make the ansantz that 

$P^* \approx P_0 + P_1 \epsilon + ... $ 

$u_{2}^* \approx U_{2,0} + U_{2,1} \epsilon + ... $ 

$u_{1}^* \approx U_{1,0} + U_{1,1} \epsilon + ... $ 

Taking all the $\epsilon^{-1}$ terms from the 2nd equation, we get

$(P_0)_{z^*} = -1$, so $P^* = C(x^*, t^*)-z^* $ (To order O(1)), With the initial condition that $P(z = h(x, t)) = P_{atm}$ so $P^* = P_{atm}+(h^*(x^*,t^*)-z^*) $ at leading order. [Since $\epsilon$ is small, the actual solution is the leading order solution plus perturbations on the order of $\epsilon$ and higher. However, this is negligible because $\epsilon$ is so small]

For all the terms without $\epsilon$ in the 2nd equation, we get

$(U_{2,0})_{t^*} = 0$, so ${u^*}_{2} = C(x^*)$ [as ${u^*}_{2z^*} = 0$, it can only be a function of x]

Going back to $u_{2x} = u_{1z}$, nondimensionalising yields

$${u^*}_{1z^*} = \frac{A}{L} {u^*}_{2x^*} =  \epsilon {u^*}_{2x^*}$$

Furthermore, we have that 

$$({u^*}_{2x^*})_{z^*} = ({u^*}_{2z^*})_{x^*} = 0$$

$$({u^*}_{2x^*})_{t^*} = ({u^*}_{2t^*})_{x^*} = 0$$

$$ \frac{A}{L}  ({u^*}_{2x^*})_{x^*} = ({u^*}_{1z^*})_{x^*} =  ({u^*}_{1x^*})_{z^*} = 0$$

Therefore,  ${u^*}_{2x^*} = c$ for some constant c. 

So 

$${u^*}_{1z^*} = \epsilon *c$$

Since $ {u^*}_{1z^*}$ is of the order $\epsilon$, at leading order, it is then 0. 

It is also true that the flux in minus the flux out (left side) should give the rate of change of the volume in the region, so

$$\int_0^{h^*} u^*_1(a, z^*, t^*) dz^* - \int_0^{h^*} u^*_1(b, z^*, t^*) dz^* = \frac{\partial}{\partial t^*}\int_a^b h^* dx^* $$ 

by the fundamental theorem of calculus, the left side is

$$\int_0^{h^*} u^*_1(a, z^*, t^*) dz^* - \int_0^{h^*} u^*_1(b, z^*, t^*) dz^* =-\int_a^b \frac{\partial}{\partial x^*} \int_0^{h^*} u^*_1(x^*, z^*, t^*) dz^* dx^*  $$ 

So then

$$-\int_a^b \frac{\partial}{\partial x^*} \int_0^{h^*} u^*_1(x^*, z^*, t^*) dz^* dx^* = \frac{\partial}{\partial t^*}\int_a^b h^* dx^* $$

$$\int_a^b ((\frac{\partial}{\partial x^*} \int_0^{h^*} u^*_1(x^*, z^*, t^*) dz^*) + h^*_{t^*} )dx^* = 0 $$

As $u_{1z^*} = 0, u_1 = f(x^*, t^*)$:

$$(\frac{\partial}{\partial x^*} \int_0^{h^*} u^*_1(x^*, z^*, t^*) dz^*) + h^*_{t^*}  = 0 $$

$$\frac{\partial}{\partial x^*} (h^*u^*_1(x^*, z^*, t^*)) + h^*_{t^*}  = 0 $$

$$ (h^*u^*_1)_{x^*} + h^*_{t^*}  = 0 $$

This is the first non-dimensional shallow water equation.


To get the second, go back to:

$$ {u^*}_{1t^*} +  P^*_{x^*}  = 0   $$

Note that $P^*_{x^*} = {h^*}_{x^*}$

$$ {u^*}_{1t^*} +  {h^*}_{x^*}  = 0   $$

$$ h^**{u^*}_{1t^*} +  h^**{h^*}_{x^*}  = 0   $$

Since $(h^*{u^*}_1)_{t^*} = h^*_t u^*_1 + {u^*}_{1t} h^*$

$$ (h^*{u^*}_1)_{t^*} - h^*_t u^*_1   +  h^**{h^*}_{x^*}  = 0   $$

$$ (h^*{u^*}_1)_{t^*} + (h^*u^*_1)_{x^*} u^*_1   +  h^**{h^*}_{x^*}  = 0   $$

Since ${u^*}_{1x} = 0$, bring it inside the derivative

$$ (h^*{u^*}_1)_{t^*} + (h^* (u^*_1)^2)_x   + h^**{h^*}_{x^*}  = 0   $$

Now note that $h^**{h^*}_{x^*} = ({\frac{1}{2}(h^*)^2})_{x^*}$

$$ (h^*{u^*}_1)_{t^*} + (h^* (u^*_1)^2)_x   + ({\frac{1}{2}(h^*)^2})_{x^*}  = 0   $$

$$ (h^*{u^*}_1)_{t^*} + (h^* (u^*_1)^2+{\frac{1}{2}(h^*)^2})_{x^*}  = 0   $$

Giving the second shallow water equation in non-dimensional form

\vspace{13cm}

asds

By loop analysis, we get the following:

$$i = i_L + i_C$$
$$V_O = Q/C = V_{in} -iR$$
$$Q/C = L \frac{di_L}{dt} + R_L i_L$$

With some math, we get the following second order equation;

$$\frac{d^2i_L}{dt^2} + \frac{L+RCR_L}{RLC} \frac{di_L}{dt} + \frac{R_L+R}{RLC} i_L = \frac{V_{in}}{RLC}$$

Therefore we take $\alpha$ and $\omega$ as: 
$$\alpha =  \frac{L+RCR_L}{2RLC} = 3129/s$$
$$\omega_0 = \sqrt{\frac{R_L+R}{RLC}} = 6025/s$$ 
$$\omega =  5149/s$$
$$\zeta = 0.52$$

As the particular solution is simply $V_{in}/(R+R_L)$, The solution is then

$$i_L = e^{-3129t} [A\cos(5149t)+B\sin(5149t)] + 8*10^{-4}$$

And the initial conditions are that $i_L(0-) = i_L(0+) = 0$, and given that $V_C(0-)  = V_C(0-) = 0$, we know

$$\frac{di_L(0)}{dt} = \frac{V_C(0)  -R_Li_L(0)}{L }  = 0 $$

$$i_L = (e^{-3129t} [-8\cos(5149t)-4.86\sin(5149t)] + 8)*10^{-4}$$

$$V_O = L \frac{di_L}{dt} + R_L i_L$$

$$V_O = 0.2(1-e^{-3129t}[cos(5149t)-3.62sin(5149t)]) $$

\vspace{5cm}

By KVL, we have

$$\frac{di}{dt}*L + (R+R_L)i + \frac{Q}{C} i = V_{in}$$

Differentiating, 

$$\frac{d^2i}{dt^2} + \frac{R+R_L}{L}\frac{di}{dt} + \frac{i}{CL} = 0$$

$$\alpha =  \frac{R+R_L}{2L} = 11466/s$$
$$\omega_0 = \sqrt{\frac{1}{LC}} = 8165/s$$ 

Since $\alpha$ is greater than $\omega_0$, it is overdamped and thus

$$i = Ae^{-3301t}+Be^{-19631t}$$

We know the initial conditions are that $i_L(0-) = 0 = i_L(0+) = i(0)$
And since
$V_C(0-) = 0 = V_C(0+)$, 
$$\frac{di(0+)}{dt} = V_{in}/L = 6.67A/s$$

$$i = 4*10^{-4} (e^{-3301t}-e^{-19631t} )$$

$$V_C = V_{in}- \frac{di}{dt}*L - (R+R_L)i$$

$$V_C = 1 - 1.20e^{-3301t}+.22e^{-19631t}$$



\vspace{10cm}


By KVL, we have

$$\frac{di}{dt}*L + (R+R_L)i + \frac{Q}{C} i = V_{in}$$

Differentiating, 

$$\frac{d^2i}{dt^2} + \frac{R+R_L}{L}\frac{di}{dt} + \frac{i}{CL} = 0$$

$$\alpha =  \frac{R+R_L}{2L} = 1160/s$$
$$\omega_0 = \sqrt{\frac{1}{LC}} = 8165/s$$ 

Since $\alpha$ is less than $\omega_0$, it is underdamped. 
$\omega = \sqrt{\omega_0^2 -\alpha^2} = 8082/s$ 

$$i = e^{-1106t}[A\cos(8082t)+B\sin(8082t)]]$$

We know the initial conditions are that $i_L(0-) = 0 = i_L(0+) = i(0)$
And since
$V_C(0-) = 0 = V_C(0+)$, 
$$\frac{di(0+)}{dt} = V_{in}/L = 6.67A/s$$

$$i = 8.25*10^{-4} (e^{-1106t} \sin(8082t) )$$

$$V_C = V_{in}- \frac{di}{dt}*L - (R+R_L)i$$

$$V_C = 1  - e^{-1106t}[\cos(8082t)+0.15\sin(8082t)]$$








\end{document}
