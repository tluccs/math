\documentclass{article}
\begin{document}
Tameez Latib

Problem 5

I, Tameez Latib, declare that this work is my own. I did this work honestly and can fully stand behind everything that I have written.

---

Let's compare the relativistic and classical kinetic energies. We know the classical kinetic energy $K = \frac{1}{2}mv^2$.
The relativistic, $K_r$ is

$$K_r = mc^2(\frac{1}{\sqrt{1-v^2/c^2)}} - 1)$$

We know the taylor series for 

$$\frac{1}{\sqrt{1+x}} = 1 - \frac{1}{2} x + \frac{3}{8} x^2 - O(x^3)$$

So substituting $x = -v^2/c^2$

$$\frac{1}{\sqrt{1-v^2/c^2}} = 1 + \frac{v^2}{2c^2}+ \frac{3v^4}{8c^4} + O(\frac{v^6}{c^6})$$

Putting this in our original equation, 

$$K_r = mc^2(1 + \frac{v^2}{2c^2}+ \frac{3v^4}{8c^4} + O(\frac{v^6}{c^6})- 1)$$

$$K_r = m(\frac{v^2}{2}+ \frac{3v^4}{8c^2} + O(\frac{v^6}{c^4}))$$

$$K_r = \frac{mv^2}{2}+ m(\frac{3v^4}{8c^2} + O(\frac{v^6}{c^4}))$$

So we see that $K_r - K$ is approximately 0 when $v << c$

In fact, for almost all everyday experiences, the fastest objects we encounter are planes, which travel around 250m/s, whereas 
$c \approx 3*10^8 m/s$

so $v^4/c^2 \approx 6.9*10^{-13} m^2/s^2$

Compared with $v^2 = 6.25*10^4 m^2/s^2$

we see that the contribution to $K_r$ from all terms other than the first- $mv^2/2$ are negligible. 

This shows that, for everyday use, $K_r = K$ when taking into account precision or significant figures 


\end{document}
