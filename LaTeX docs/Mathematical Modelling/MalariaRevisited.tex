\documentclass{article}
\begin{document}
Tameez Latib

Problem 9

I, Tameez Latib, declare that this work is my own. I did this work honestly and can fully stand behind everything that I have written.

---

From the previous malaria model, we had 

$$H' = IB \gamma \frac{P-H}{P} - \rho H$$

$$I' = - \mu I + (M-I)B \frac{H}{P}$$

Now, given the fact that $\rho = \mu$ and $P = M$ (in values)

Let's non-dimensionalize this, by defining a few variables:

$H = h\bar{h}$

$I = m\bar{i}$

$t = \tau\bar{t}$

Then we have

$$\frac{\bar{h}}{\bar{t}} h' = m\bar{i} B \gamma \frac{P-h\bar{h}}{P} - \rho h\bar{h}$$

$$ h' = m\bar{i}\bar{t} B \gamma \frac{P-h\bar{h}}{P \bar{h}} - \rho h\bar{t}$$

Setting $\bar{t} = \frac{1}{\rho}$, $\bar{h} = P$,  

$$ h' = m\bar{i}\bar{t} B \gamma \frac{1-h}{P} - h$$

And 

$$\frac{\bar{i}}{\bar{t}} m' = - \mu m\bar{i} + (M-m\bar{i})B \frac{h\bar{h}}{P}$$

$$ m' = - \mu m\bar{t} + (M-m\bar{i})B \frac{h\bar{h}\bar{t}}{P\bar{i}}$$

Since $\mu = \rho$, $\bar{t} = \frac{1}{\mu}$. Now set $\bar{i} = M$

 $$ m' = - m + (1-m)B{h\bar{t}}$$
 
 Now let $B \bar{t} = \epsilon$, as $\epsilon$ is small. Then
 
  $$ m' = - m + \epsilon (1-m)h$$
  
  $$ h' = m\bar{t} B \gamma \frac{(1-h)M}{P} - h$$
  
  $M/P = 1$, so this simplifies to 
  
  $$ h' = m \epsilon \gamma (1-h) - h$$
 
Now, lets make the ansantz that 

$h = h_0 + \epsilon h_1 + ... $

$m = m_0 + \epsilon m_1 + ... $

As $\epsilon < 1$ and the terms on the order of $\epsilon^2$ or higher are negligible. 

We can now solve the differential equation subject to the constraints of 

$h(0) = \zeta$

$m(0) = \eta$

Under these constraints, all order epsilon and higher terms are 0 at $\tau = 0$ and the first order terms are

$h_0(0) = \zeta$

$m_0(0) = \eta$

The ansantz creates the following differential equations: 

  $$ h_0' =  - h_0$$
  $$ m_0' =  - m_0$$
  $$ h_1' = m_0  \gamma (1-h_0) - h_1$$
  $$ m_1' = - m_1 +(1-m_0)h_0$$
  
  The first two are simple,
  
    $$ h_0 =  \zeta * e^{-\tau}$$
    $$ m_0 =  \eta * e^{-\tau}$$
    
  Then 
  
    $$ h_1' + h_1 = m_0  \gamma (1-h_0)$$
    $$ (h_1e^{\tau})' = \eta \gamma (1 -  \zeta * e^{-\tau})$$
    $$ (h_1e^{\tau}) = \eta \gamma (\tau +  \zeta * e^{-\tau}) + C$$
    $$ h_1 = \eta \gamma (\tau e^{-\tau} +  \zeta * e^{-2\tau}) + Ce^{-\tau}$$
    $$h_1(0) = \eta \gamma ( \zeta ) + C = 0$$
    
    So finally: 
  
      $$ h_1 = \eta \gamma (\tau e^{-\tau} +  \zeta * e^{-2\tau} - \zeta * e^{-\tau})$$
  
  Similarly
  
    $$ m_1' + m_1 = h_0  (1-m_0)$$
    $$ (m_1e^{\tau})' = \zeta (1 -  \eta * e^{-\tau})$$
    $$ (m_1e^{\tau}) = \zeta  (\tau +  \eta * e^{-\tau}) + C$$
    $$ m_1 = \zeta (\tau e^{-\tau} +  \eta * e^{-2\tau} - \eta * e^{-\tau}) $$
    
So up to order $\epsilon$,

$$h = \zeta * e^{-\tau} + \epsilon \eta \gamma (\tau e^{-\tau} +  \zeta * e^{-2\tau} - \zeta * e^{-\tau})$$

$$m = \eta * e^{-\tau} + \epsilon \zeta  (\tau e^{-\tau} +  \eta * e^{-2\tau}- \eta * e^{-\tau})$$
  
\end{document}
