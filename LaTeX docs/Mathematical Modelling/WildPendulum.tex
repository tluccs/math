\documentclass{article}
\begin{document}
Tameez Latib

Problem 10

I, Tameez Latib, declare that this work is my own. I did this work honestly and can fully stand behind everything that I have written.

---

Given a pendulum with the following information:
Thin rod, length L = 1m; mass of rod is negligible, mass at pivot point m = 1kg, at t = 0, $\theta$ = 0, and 400 J Kinetic Energy is imparted into the pendulum.

We know $E = 1/2 * mv^2$ 

Therefore $v = \sqrt{(2E/m)}$

Also, $-mgL sin(\theta) = torque_{net} = I \theta''$ 

As the net force is due to gravity, and taking the magnitude of the cross product of r and f (to get torque), we get the left hand side.

Here, $I = mL^2,$ 

So our differential equation is

Also, $-g/L*sin(\theta) = \theta''$ 

From $v = \sqrt{(2E/m)}$, this tells us $\theta'(0) = 1/L*\sqrt{(2E/m)} = \sqrt{\frac{2E}{mL^2}}$ 

Let's nondimensionalise the model, by letting

$$t = \tau*\bar{t} = \tau*\sqrt{\frac{mL^2}{2E}}$$

I.e. $\theta'(0) = 1/\bar{t} $

And note $\theta$ is already unit-less, so we have 

$$-\frac{g}{L}*sin(\theta) = \theta'' / \bar{t}^2$$

$$-\frac{g \bar{t}^2}{L}*sin(\theta) = \theta''$$

Let $\epsilon = \frac{g \bar{t}^2}{L}$, so

$$-\epsilon*sin(\theta) = \theta''$$

Since $\epsilon < 1$, we make the ansantz that $\theta = \theta_0 + \epsilon*\theta_1 ... $

Furthermore, note the taylor expansion:

$$sin(\theta) = x - x^3/6 + ...$$

So we have 

$$0 = \theta_0''$$

Subject to $\theta_0(0) = 0$, $\theta_0'(0) = 1$ (Since $\theta_0'(t=0) = 1/\bar{t}$). Then it is obvious that

$$\theta_0 = \tau $$

And our other differential equation is:

$$-\epsilon(\theta_0 + O(\epsilon) - (\theta_0 + O(\epsilon) )^3/6 + ... ) = \epsilon*\theta_1 ''$$

Grouping all the O(1) terms, 

$$(\theta_0) - (\theta_0) )^3/6 + ... ) = \theta_1 ''$$

And the left side is simply the taylor series for sin, so

$$-sin(\theta_0) = \theta_1 ''$$

$$\theta_1 '' = -sin(\tau)$$

$$\theta_1 = sin(\tau)+C\tau+D$$

Given that  $\theta_1(0) = 0$, $\theta_1'(0) = 0$

Then D = 0 and C = -1 

So the final solution is 

$$\theta = \tau + \epsilon(sin(\tau)-\tau)$$







  
\end{document}
