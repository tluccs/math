\documentclass{article}
\usepackage{graphicx}
\usepackage{amsmath}
\begin{document}
Tameez Latib

Problem 13

I, Tameez Latib, declare that this work is my own. I did this work honestly and can fully stand behind everything that I have written.

----

Suppose that osteoblasts secrete uncalcified bone matrix from  x = 0 at a constant
rate, so that all parts of the matrix move at speed s > 0 (along the +x direction)
Let?s suppose the crystals are uniformly deposited and spaced, have a spherical shape, and
grow at a rate proportional to their surface area.

From this, let's model the Volume (of a spherical crystal) as a function of x and t, V(x, t). With V' (the growth) being proportional to surface area 

$$(V(x, t))' \propto 4\pi r^2$$ 

Noting that $V(x,t) = 4/3 \pi r^3$, Then $V^{2/3} \propto r^2$

Hence

$$(V(x, t))' \propto V^{2/3}$$ 

$$(V(x, t))' = c*V^{2/3}$$ 

Where c is a constant. We can also model the position x as a function of t; $x(t) = st + x(0)$
That is, the crystals move at a fixed speed s from their initial position x(0). 

Now, by the chain rule, the left hand side is:

$$V' = V_t t' + V_x x' = V_t + sV_x$$

So now we have that 

$$V_t + sV_x = cV^{2/3}$$

to non-dimensionalise, 

$V = v^*\bar{v}$
$t = t^*\bar{t}$
$x = x^*\bar{x}$

Then:

$$\frac{\bar{v}}{\bar{t}}v^*_{t^*} + \frac{\bar{v}s}{\bar{x}}v^*_{x^*} = c\bar{v}^{2/3}{v^*}^{2/3}$$

Letting $\bar{x} = s\bar{t}$, $\bar{v} = (c\bar{t})^3$,

$$v^*_{t^*} + \frac{\bar{t}s}{\bar{x}}v^*_{x^*} = c\frac{\bar{t}}{\bar{v}^{1/3}}{v^*}^{2/3}$$

We obtain:

$$v^*_{t^*} + v^*_{x^*} = {v^*}^{2/3}$$

For convenience, I will omit the '*'s, since there will be no further non-dimensionalising or converting back to dimensional form

Now to solve
$$v_{t} + v_{x} = {v}^{2/3}$$

To the constraints: $v(x, 0) = 0, v(0, t) = 1$

First, let $x = x(\tau, \sigma)$, $t = t(\tau, \sigma)$, with $\sigma$ parametrizing the boundary. Then

$$\frac{dv}{d\tau}= v_{x}\frac{\partial x}{\partial \tau} + v_{t}\frac{\partial t}{\partial \tau}$$

If we let 

$$\frac{\partial x}{\partial \tau} = \frac{\partial t}{\partial \tau} = 1$$

Then 

$$\frac{dv}{d\tau} = v^{2/3}$$

This differential equation can be solved, to get

$$v = \frac{(\tau+c)^3}{27}$$

Now to convert back, $\sigma$ is the boundary, so 

\begin{equation*}
x(0, \sigma) = 
\begin{cases}
0 & \text{if $\sigma < 0$}\\
\sigma & \text{if $\sigma > 0$}
\end{cases}
\end{equation*}

\begin{equation*}
t(0, \sigma) = 
\begin{cases}
-\sigma & \text{if $\sigma < 0$}\\
0 & \text{if $\sigma > 0$}
\end{cases}
\end{equation*}

Then we have that $x - t = \sigma$. Furthermore, since $\frac{\partial x}{\partial \tau} = 1$, $x = \tau + x(0, \sigma)$

\begin{equation*}
\tau = x - x(0, \sigma) = 
\begin{cases}
x -0 & \text{if $\sigma < 0$}\\
x - \sigma & \text{if $\sigma > 0$}
\end{cases}
\end{equation*}

\begin{equation*}
\tau = 
\begin{cases}
x & \text{if $x < t$}\\
t & \text{if $x > t$}
\end{cases}
\end{equation*}

Then we have

\begin{equation*}
v(x,t) = 
\begin{cases}
{(x+c)^3}/{27} & \text{if $x < t$}\\
{(t+c)^3}/{27} & \text{if $x > t$}
\end{cases}
\end{equation*}

For the initial conditions, v(0, t) = 1  and v(x, 0) = 0, This means (as $x = 0 < t$ and $t = 0 < x$ )

\begin{equation*}
v(x,t) = 
\begin{cases}
{(x+3)^3}/{27} & \text{if $x < t$}\\
{(t)^3}/{27} & \text{if $x > t$}
\end{cases}
\end{equation*}

However, x > t means that we are trying to find the volume of a crystal where there is no matrix. So the only solution should be 

$$v(x, t) = \frac{(x+3)^3}{27}$$

\end{document}
