\documentclass{article}
\usepackage{graphicx}
\usepackage{amsmath}
\begin{document}
Tameez Latib

Problem 17

I, Tameez Latib, declare that this work is my own. I did this work honestly and can fully stand behind everything that I have written.

----

Let's consider a model of Temperature to show that wind 'chills' objects, when the ambient temperature is cold. 
Suppose that the ambient air temperature is $T_C$ and consider a person/object with temperature $T_H$. 

This gives the initial conditions (person starts at x = 0)

$T(0) = T_H$

$T(\inf) = T_C$

Let $v$ be the velocity of the wind. Since it comes from $x = \inf$ and moves to $x = 0$ then $v < 0$

Also, let D be the thermal diffusivity. and $\sigma$ be the thermal conductivity of air. 

By the heat equation

$$T_t - D\bigtriangleup T = 0$$


But note that in one dimension,

$$\bigtriangleup T = T_{xx} $$

And

$$ \frac{\partial T}{\partial t} =   \frac{\partial T}{\partial x}  \frac{dx}{dt} =  T_xv$$  

So then the equation becomes

$$vT_x - DT_{xx} = 0$$

Letting $T_x = q$

$$vq - Dq_x = 0$$

$$q = ae^{vx/D}$$

$$T_x = q$$

$$T = be^{vx/D} + c$$

With the initial conditions, 

$T(\inf) = c = T_C$ (as v is negative)

$T(0) = b + c = b + T_C =  T_H$, $b = T_H - T_C$

Then 

$$T = (T_H-T_C)e^{vx/D} + T_C$$

Since v is negative, as the speed $|v|$ goes up, v decreases. Therefore, $e^{vx/D}$ decreases (leaving x, D constant). And since $T_H - T_C > 0$  (Object temperature is greater than ambient temperature), we have that T decreases. 

Therefore, given the same x value, increasing the speed means decreasing the temperature at that value. Hence the larger the wind speed, the colder it feels. This makes sense, as on a cold day, it 'feels' a lot colder when it is very windy than when it is not. Because the icy winds transfer heat.

\end{document}
