\documentclass{article}
\usepackage{graphicx}
\usepackage{amsmath}
\begin{document}
Tameez Latib


----

1. Let 

$$f(x) = x^2 - 0.7x$$

a. Note 

$f(-1) = 1.7$

$f(0.5) = -0.1$

$f(1) = 0.3$

Since $f$ is continuous, the intermediate value theorem implies there exists a zero on the interval (-1, 0.5) and there exists a zero on the interval (0.5, 1). Furthermore, by the fundamental theorem of algebra, $f$ only has 2 zeros. Therefore, there is exactly one zero on the interval (0.5, 1).


b. With the bisection method, the error of the nth point $p_n$, $E_n$ is given by 

$$E_n = |p_n - p| \le \frac{b-a}{2^n} $$

So for $a = 0.5, b = 1$, we have 

$$E_n \le \frac{1}{2^{n+1}} \le 10^{-5} $$

Which gives us a minimum value of n of $n = 16$

---

2. if $f(x)$ is continuous on $I = [a, b]$ and $f(x) \in \forall x \in I$, then there exists $c$ such that $f(c) = c \in I$

Proof:  Consider the function $g(x) = f(x) - x$. $f$ has a fixed point at $c$ iff g has a root $c$.

Now consider the values at the endpoints,

$g(a) = f(a) - a \ge a - a = 0$ 

$g(b) = f(b) - b \le b - b = 0$

We can assume $g(a) \ne 0 \ne g(b)$ otherwise take the endpoint(s) as the fixed point. Therefore $g(b) < 0 < g(a)$ 

So by the intermediate value theorem value theorem, $g$ has a zero in the interval $I$. Therefore $f$ has a fixed point in $I$

--

3.a. Given $p_0 = 3$, 

$p_{n+1} = \frac{p_n^2+3}{2p_n}$

Then we find 

$p_1 =\frac{3^2+3}{2*3} = 2$

$p_2 =\frac{2^2+3}{2*2} = \frac{7}{4}$ 

b. Find all possible limits of $p_n$:

If $p$ is a limit of the sequence, then we must have 


$$\lim_{n \to \inf} p_{n+1} = \lim_{n \to \inf} \frac{p_n^2+3}{2p_n}$$

$$p =  \frac{p^2+3}{2p}$$

$$p^2 = 3$$

$$p = \pm \sqrt{3}$$

So the only limits of the sequence are $ \sqrt{3}, -\sqrt{3}$

c. If we apply Newtons method to $f(x) = x^2 - 3$, the iterative step to produce the next point in the sequence is given by: 

$$p_{n+1} = p_n - \frac{f(p_n)}{f'(p_n)}$$

$$p_{n+1} = p_n - \frac{p_n^2 - 3}{2p_n}$$

$$p_{n+1} = \frac{2p_n^2-p_n^2 + 3}{2p_n}$$

$$p_{n+1} = \frac{p_n^2 + 3}{2p_n}$$

Which is the given sequence. Therefore the given sequence is precisely the sequence generated by Newton's method to find the zeros of $f(x)$

--

4.
Let $$f(x) := x^2 - 3$$ 

We want to find the zeroes of $f$ on $I =[0, 4]$

a. With the secant method,

$$p_{n+2} = p_{n+1} - \frac{f(p_{n+1}) *(p_{n+1}-p_n)}{f(p_{n+1}) - f(p_n)}$$

So for starting points:  $p_0 =0.5, p_1 = 3$ we have that:

$$p_2 = p_{1} - \frac{f(p_{1}) *(p_{1}-p_0)}{f(p_{1}) - f(p_0)}$$
$$p_2 = 3 - \frac{6 *(3-0.5)}{6 - -2.75} = 1.286$$
$$p_3 = 1.286 - \frac{-1.346 *(1.286-3)}{-1.346 - 6} = 1.6$$

b. With the method of false position,

We compute 

$$c = b - \frac{f(b) *(b-a)}{f(b) - f(a)}$$

Then update $a = c$ if $f(c)f(b) \le 0$ else $b = c$

so with $a = p_0 = 0.5, b = p_1 = 3$

$$c = b - \frac{f(b) *(b-a)}{f(b) - f(a)} = 1.286 = p_2$$
$f(c)*f(b) \le 0$ So this means we update $a = c$, and find our new c: 
$$c = 3 - \frac{6 *(3-1.286)}{6 - -1.346} = 1.6 = p_3$$

This gives us the same $p_2, p_3$ as a.

5. Refer to PDF of code + outputs







\end{document}
