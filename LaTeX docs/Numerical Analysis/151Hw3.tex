\documentclass{article}
\usepackage{graphicx}
\usepackage{amsmath}
\begin{document}
Tameez Latib


----

1. 

$f(x) = 1/x, x_i = i + 1, 0 \le i \le 2$, find Lagrange polynomial interpolating $(x_i, f(x_i))$ by

a. Lagrange formula:

We define 

$$L_0(x) = \frac{(x-2)(x-3)}{(1-2)(1-3)}$$

$$L_1(x) = \frac{(x-1)(x-3)}{(2-1)(2-3)}$$

$$L_2(x) = \frac{(x-1)(x-2)}{(3-1)(3-2)}$$

And our solution is 

$$P(x) = L_0(x)*f(x_0) + L_1(x)*f(x_1)  + L_2(x)*f(x_2)$$

$$P(x) = \frac{11}{6} - x + \frac{1}{6}x^2$$

And to check, $P(1) = 1, P(2) = 1/2, P(3) = 1/3$ So therefore this is the unique $P(x)$ corresponding to the interpolation


b. Now using Nevilles method: 

We define the following polynomials, through an iterative process, where $$P_{n_1, n_2, ... n_m}(x_j) = f(x_j),  j \in \{n_1, n_2, ... n_m\} $$

$$P_{0}(x) = 1$$
$$P_{1}(x) = \frac{1}{2}$$
$$P_{2}(x) = \frac{1}{3}$$

$$P_{0,1}(x) = \frac{P_{1}(x)(x-1) - P_{0}(x)(x-2) }{2-1} = \frac{3-x}{2}$$
$$P_{1,2}(x) = \frac{P_{2}(x)(x-2) - P_{1}(x)(x-3) }{3-2} = \frac{5-x}{6}$$

$$P_{1,2,3}(x) = \frac{P_{1,2}(x)(x-1) - P_{0,1}(x)(x-3) }{3-1} = \frac{11}{6} - x + \frac{1}{6}x^2$$

And this is the same $P$ as before

c. Now using divided differences:

$$f[x_0] = 1$$
$$f[x_1] = \frac{1}{2}$$
$$f[x_2] = \frac{1}{3}$$

$$f[x_0, x_1] = \frac{f[x_1] - f[x_0]}{2-1} = \frac{-1}{2}$$
$$f[x_1, x_2] =  \frac{f[x_2] - f[x_1]}{3-2} = \frac{-1}{6}$$

$$f[x_0, x_1, x_2] =  \frac{f[x_1, x_2]- [x_0, x_1]}{3-1} = \frac{1}{6}$$

And our $P$ is now

$$P(x) = f[x_0] + f[x_0, x_1] (x-1)+ f[x_0, x_1, x_2] (x-1) (x-2) = \frac{11}{6} - x + \frac{1}{6}x^2$$

Which is again the same polynomial.

2. Find the natural cubic spline passing through $(-1, 1) , (0, 1), (1, 2)$

Let 
$$S_1(x) =a_1+b_1(x+1)+c_1(x+1)^2+d_1(x+1)^3$$
$$S_2(x) =a_2+b_2(x)+c_2(x)^2+d_2(x)^3$$

By the endpoint conditions, 

$S_1(-1) = 1 \implies a_1 = 1,$

$S_2(0) = 1 \implies a_2 = 1,$

$S_1(0) = 1 \implies b_1+c_1+d_1 = 0,$

$S_2(1) = 2 \implies  b_2+c_2+d_2 = 1$

Because it is natural, we have 

$S_1''(-1) = 0 \implies c_1 = 0,$

$S_2''(1) = 0 \implies 2c_2 + 6d_2= 0$

And the continuity of first/second derivatives give us the conditions

$S_1'(0) = S_2'(0) \implies b_1 + 2c_1 + 3d_1 = b_2,$

$S_1''(0) = S_2''(0) \implies 2c_1 + 6d_1 = 2c_2$

Doing some algebra we get 

$d_1 = c_2/3,  d_2 = -c_2/3, b_1 = -c_2/3, b_2 = 2c_2/3$

So then 

$c_2 = 3/4$

Our final solution is 

$$S_1(x) = 1-\frac{1}{4}(x+1)+\frac{1}{4}(x+1)^3$$
$$S_2(x) = 1+\frac{1}{2}(x)+\frac{3}{4}(x)^2-\frac{1}{4}(x)^3$$


\[ S(x) = \begin{cases} 
      S_1(x) & -1\leq x\leq 0\\
      S_2(x) & 0\leq x\leq 1
   \end{cases}
\]

With $S(x)$ being our cubic spline

3. Prove the following theorem: Let $f \in C^1([a, b])$ and $x_0, x_1, ...x_n$ be n distinct nodes in $[a, b]$, and let

$$H(x)= \sum_{i=0}^n ( f(x_i)H_{n,i}(x) + f'(x_i )\hat{H}_{n,i}(x) )$$

where 

$$H_{n,j}(x) = (1-2(x-x_j)L'_{n,j}(x_j))L^2_{n,j}(x)$$

$$\hat{H}_{n,j}(x) = (x-x_j)L^2_{n,j}(x)$$

Then $H(x_i) = f(x_i), H'(x_i) = f'(x_i)$ for $0 \le i \le n$

Note that if we plug in $x_i \in \{x_0, x_1, ...x_n\}$, 

$$H_{n,j}(x_i) = (1-2(x_i-x_j)L'_{n,j}(x_j))L^2_{n,j}(x_i) = \delta_{i,j}$$

$$\hat{H}_{n,j}(x_i) = (x_i-x_j)L^2_{n,j}(x_i) = 0$$

Similarly, if we plug in $x_i$ to these functions derivatives, 

$$H_{n,j}'(x_i) = (-2L'_{n,j}(x_j))L^2_{n,j}(x_i) + 2L'_{n,j}(x_i)*(1-2(x_i-x_j)L'_{n,j}(x_j))  = 0$$

$$\hat{H}_{n,j}'(x_i) = L^2_{n,j}(x_i)+2(x_i-x_j)L'_{n,j}(x_i) = \delta_{i,j}$$

Therefore 

$$H(x_j) =  \sum_{i=0}^n ( f(x_i)H_{n,i}(x_j) + f'(x_i )\hat{H}_{n,i}(x_j) ) $$
$$ = \sum_{i=0}^n f(x_i)\delta_{i,j}  = f(x_j)$$


$$H'(x_j) =  \sum_{i=0}^n ( f(x_i)H_{n,i}'(x_j) + f'(x_i )\hat{H}_{n,i}'(x_j) ) $$
$$ = \sum_{i=0}^n f'(x_i)\delta_{i,j}  = f'(x_j)$$

4.  Let $f \in C^3([x_0-h, x_0+h])$ 

4.a: $P(x)$, the lagrange polynomial interpolating nodes $x_0-h, x_0, x_0+h$

Define 

$$ L_{-}(x) = \frac{(x-x_0)(x-x_0-h)}{(x_0-h-x_0)(x_0-h-x_0-h)} = \frac{(x-x_0)(x-x_0-h)}{2h^2} $$
$$ L_{0}(x) = \frac{(x-x_0+h)(x-x_0-h)}{(x_0-x_0+h)(x_0-x_0-h)} =  \frac{(x-x_0+h)(x-x_0-h)}{-h^2} $$
$$ L_{+}(x) = \frac{(x-x_0)(x-x_0+h)}{(x_0+h-x_0)(x_0+h-x_0+h)} = \frac{(x-x_0)(x-x_0+h)}{2h^2} $$

So then 

$$P(x) = L_{-}(x)f(x_0-h) + L_0(x)f(x_0) + L_{+}(x)f(x_0+h)$$

4.b

$$E(x) = \frac{f^{(3)}(\xi(x))}{6}(x-x_0+h)(x-x_0)(x-x_0-h)$$ 

By the lagrange interpolation theorem, where $\xi(x) \in [x_0-h, x_0+h]$ 

4.c

$$f'(x_0) = P'(x_0) + E'(x_0)$$

Note that 

$$ L_{-}'(x_0) = \frac{-h}{2h^2} = \frac{-1}{2h} $$
$$ L_{0}'(x_0) =  0 $$
$$ L_{+}'(x_0) = \frac{h}{2h^2} = \frac{1}{2h} $$

$$f'(x_0) = \frac{-f(x_0-h) + f(x_0+h)}{2h} - \frac{h^2f^{(3)}(\xi(x))}{6} $$
where $\xi(x) \in [x_0-h, x_0+h]$ 

4.d 

if $f$ is a polynomial of degree less than or equal to 2, then $f^{(3)}(x) = 0 \implies E(x) = 0$. In other words,

$$f(x) = P(x)$$

and so their derivatives must also be equal;

$$f'(x) = P'(x)$$

4.e 

$$f'(x_0) - P'(x_0) = E'(x_0) = - \frac{h^2f^{(3)}(\xi(x))}{6} $$
where $\xi(x) \in [x_0-h, x_0+h]$ 


5. Refer to code
 




\end{document}
